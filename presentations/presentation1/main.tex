\documentclass[ucs,9pt]{beamer}

% Copyright 2004 by Till Tantau <tantau@users.sourceforge.net>.
%
% In principle, this file can be redistributed and/or modified under
% the terms of the GNU Public License, version 2.
%
% However, this file is supposed to be a template to be modified
% for your own needs. For this reason, if you use this file as a
% template and not specifically distribute it as part of a another
% package/program, I grant the extra permission to freely copy and
% modify this file as you see fit and even to delete this copyright
% notice.
%
% Modified by Tobias G. Pfeiffer <tobias.pfeiffer@math.fu-berlin.de>
% to show usage of some features specific to the FU Berlin template.

% remove this line and the "ucs" option to the documentclass when your editor is not utf8-capable
\usepackage[utf8x]{inputenc}    % to make utf-8 input possible
\usepackage[english]{babel}     % hyphenation etc., alternatively use 'german' as parameter

\include{fu-beamer-template}  % THIS is the line that includes the FU template!

\usepackage{arev,t1enc} % looks nicer than the standard sans-serif font
% if you experience problems, comment out the line above and change
% the documentclass option "9pt" to "10pt"

% image to be shown on the title page (without file extension, should be pdf or png)
\titleimage{fu_500}

\title{XML-Technologien SS14}

\subtitle{Projektvorstellung: Android-App}

\author{Projektgruppe 5}
% - Give the names in the same order as the appear in the paper.

\institute{Freie Universität Berlin}
% - Keep it simple, no one is interested in your street address.

\date{28. Mai 2014}
% - Either use conference name or its abbreviation.
% - Not really informative to the audience, more for people (including
%   yourself) who are reading the slides online

%\subject{Theoretical Computer Science}
% This is only inserted into the PDF information catalog. Can be left
% out.

% you can redefine the text shown in the footline. use a combination of
% \insertshortauthor, \insertshortinstitute, \insertshorttitle, \insertshortdate, ...
\renewcommand{\footlinetext}{\insertshortinstitute, \insertshorttitle, \insertshortdate}

% Delete this, if you do not want the table of contents to pop up at
% the beginning of each subsection:
\AtBeginSubsection[]
{
  \begin{frame}<beamer>{Outline}
    \tableofcontents[currentsection,currentsubsection]
  \end{frame}
}

\begin{document}

\begin{frame}[plain]
  \titlepage
\end{frame}

\section{Gruppenmitglieder}

\begin{frame}{Gruppenmitglieder}
	\begin{itemize}
	\item Martin Görick
	\item Tay Ho
	\item Ahmet-Serdar Karakaya
	\item Moritz Maxeiner
	\item Florian Mercks
	\item Andre Plötze
	\item Frank Zechert
	\end{itemize}
\end{frame}

\section{Projektidee}

\begin{frame}{Projektidee}
	\begin{itemize}
  		\item Datensatz Berlinische Galerie Landesmuseum für Moderne Kunst, Fotografie und Architektur\\[1cm]
	\end{itemize}
	\includegraphics[scale=0.5]{bild1.png}  
	\vspace*{1cm}
	\includegraphics[scale=0.5]{bild2.png} 
	\vspace*{1cm} 
	\includegraphics[scale=0.5]{bild3.png}  
\end{frame}

\begin{frame}{Android App}
	\begin{columns}[T]
		\begin{column}{.6\textwidth}
			\begin{itemize}
				\item<1-> Android-App lokalisiert den Benutzer mittels GPS in
				          regelmäßigen Abständen
				\item<2-> Position wird an unseren Server übermittelt
				\item<3-> Der Server sucht am Nutzerstandort nach alten Bildern von 
						  Gebäuden in der Nähe
				\item<4-> Gefundene Bilder werden dem Nutzer angezeigt
			\end{itemize}
		\end{column}
		\begin{column}{.4\textwidth}
			\includegraphics<1-2>[width=.7\textwidth]{map.png}
			\includegraphics<3>[width=.7\textwidth]{server.png}
			\includegraphics<4>[width=.7\textwidth]{map2.png}
		\end{column}
	\end{columns}
\end{frame}

\end{document}
